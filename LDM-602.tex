\documentclass[SE,toc,lsstdraft]{lsstdoc}

% We use commands to make it easy to find where parameter names and units
% are defined in the tables, and to allow hyphenation.
\newcommand{\paramname}[1]{\hspace{0pt}#1}
\newcommand{\unitname}[1]{\hspace{0pt}#1}

\setcounter{secnumdepth}{5}

%% Retrieve date and model version
\setDocUpstreamLocation{MagicDraw SysML}
\setDocUpstreamVersion{80}

\date{2018-01-19}

%% Allow arbitrary latex to be inserted at the end of the document.
%% Define a new version of this command in metadata.tex. It will
%% be run before the references are displayed.
\newcommand{\addendum}{}

%% Define the document title, authors, handle, and change record
\input metadata.tex

% Environment for displaying the parameter tables in
% a consistent manner. No arguments as there are no
% captions or labels.
\newenvironment{parameters}[0]{%
\setlength\LTleft{0pt}
\setlength\LTright{\fill}
\begin{small}
\begin{longtable}[]{|p{0.49\textwidth}|l|p{0.6in}|p{1.70in}@{}|}

\hline \textbf{Description} & \textbf{Value} & \textbf{Unit} & \textbf{Name} \\ \hline
\endhead

\hline \multicolumn{4}{r}{\emph{Continued on next page}} \\
\endfoot

\hline\hline
\endlastfoot
}{%
\hline
\end{longtable}
\end{small}
}

\begin{document}
\maketitle

\section{\citeds{LSE-61} Flowdown}

This section contains direct copies of requirements from \citeds{LSE-61}, based on the Level 1 requirements defined in \citeds{LDM-148}.

\subsection{Data Products}

\subsubsection{General Considerations}

\paragraph{Measurements in catalogs}\hfill  % Force subsequent text onto new line

\label{DMS-L1-REQ-0044}
\textbf{ID:} DMS-L1-REQ-0044 (Priority: 1b)

\textbf{Specification: }All catalogs shall record source measurements in flux units.

\textbf{Discussion: }Difference measurements can go negative and in multi-epoch surveys averaging of fluxes rather than magnitudes is required. This requirement does not preclude making magnitudes available where appropriate.

\emph{Derived from Requirements:}

DMS-REQ-0347:
Measurements in catalogs \newline

\paragraph{Computing Derived Quantities}\hfill  % Force subsequent text onto new line

\label{DMS-L1-REQ-0042}
\textbf{ID:} DMS-L1-REQ-0042 (Priority: 1b)

\textbf{Specification:} Common derived quantities shall be made available to end-users by either providing pre-computed columns or providing functions that can be used dynamically in queries. These should at least include the ability to calculate the reduced chi-squared of fitted models and make it as easy as possible to calculate color-color diagrams.

\textbf{Discussion:} Example quantities include those used to assess model fit quality or those required for calculating color-magnitude diagrams. Care should be taken to name the derived columns in a clear unambiguous way.

\emph{Derived from Requirements:}

DMS-REQ-0331:
Computing Derived Quantities \newline

\paragraph{Maximum Likelihood Values and Covariances}\hfill  % Force subsequent text onto new line

\label{DMS-L1-REQ-0043}
\textbf{ID:} DMS-L1-REQ-0043 (Priority: 1b)

\textbf{Specification:} Quantities delivered by all measurement algorithms shall include maximum likelihood values and covariances.

\textbf{Discussion:} Algorithms for which such values are impossible, will be documented explicitly to declare that the values are unavailable.

\emph{Derived from Requirements:}

DMS-REQ-0333:
Maximum Likelihood Values and Covariances \newline

\subsubsection{Level 1 Data Products}

\paragraph{Nightly Data Accessible Within 24 hrs}\hfill  % Force subsequent text onto new line

\label{DMS-L1-REQ-0002}
\textbf{ID:} DMS-L1-REQ-0002 (Priority: 1b)

\textbf{Specification:} With the exception of alerts and Solar System Objects, all Level 1 Data Products shall be made public within time \textbf{L1PublicT} (LSR-REQ-0104) of the acquisition of the raw image data. Alerts shall be made available within time \textbf{OTT1} (LSR-REQ-0101) from the conclusion of readout of the raw exposures used to generate each alert to the distribution of the alert to community distribution mechanisms.  Solar System Object orbits shall, on average, be calculated before the following night's observing finishes and the results shall be made available within time \textbf{L1PublicT} of those calculations being completed.

\textbf{Discussion:} Because of the processing flow of SSObject orbit determination, meeting the base 24-hours-after-data-acquisition requirement would be far more challenging than for the other L1 Data Products, but the system throughput has to be good enough such that a back log can not build up.

\emph{Derived from Requirements:}

DMS-REQ-0004:
Nightly Data Accessible Within 24 hrs \newline

\paragraph{Processed Visit Images}\hfill  % Force subsequent text onto new line

\label{DMS-L1-REQ-0007}
\textbf{ID:} DMS-L1-REQ-0007 (Priority: 1a)

\textbf{Specification: }The DMS shall produce Processed Visit Images, in which the corresponding raw sensor array data has been trimmed of overscan and corrected for instrumental signature. Images obtained in pairs during a standard visit are combined.

\textbf{Discussion:} Processed science exposures are not archived, and are retained for only a limited time to facilitate down-stream processing. They will be re-generated for users on-demand using the latest processing software and calibrations.

This aspect of the processing for Special Programs data is specific to each program.

\emph{Derived from Requirements:}

DMS-REQ-0069:
Processed Visit Images \newline

\subparagraph{Background Model Calculation}\hfill  % Force subsequent text onto new line

\label{DMS-L1-REQ-0008}
\textbf{ID:} DMS-L1-REQ-0008 (Priority: 1b)

\textbf{Specification: }The DMS shall derive and persist a background model (both due to night sky and astrophysical) for each visit image, per CCD.

\emph{Derived from Requirements:}

DMS-REQ-0327:
Background Model Calculation \newline

\subparagraph{Generate Photometric Zeropoint for Visit Image}\hfill  % Force subsequent text onto new line

\label{DMS-L1-REQ-0009}
\textbf{ID:} DMS-L1-REQ-0009 (Priority: 1b)

\textbf{Specification:} The DMS shall derive and persist a photometric zeropoint for each visit image, per CCD.

\emph{Derived from Requirements:}

DMS-REQ-0029:
Generate Photometric Zeropoint for Visit Image \newline

\subparagraph{Generate PSF for Visit Images}\hfill  % Force subsequent text onto new line

\label{DMS-L1-REQ-0010}
\textbf{ID:} DMS-L1-REQ-0010 (Priority: 1b)

\textbf{Specification:} The DMS shall determine a characterization of the PSF for any specified location in Processed Visit Images.

\emph{Derived from Requirements:}

DMS-REQ-0070:
Generate PSF for Visit Images \newline

\subparagraph{Generate WCS for Visit Images}\hfill  % Force subsequent text onto new line

\label{DMS-L1-REQ-0011}
\textbf{ID:} DMS-L1-REQ-0011 (Priority: 1a)

\textbf{Specification:} The DMS shall generate and persist a WCS for each visit image.  The absolute accuracy of the WCS shall be at least \textbf{astrometricAccuracy} in all areas of the image, provided that there are at least \textbf{astrometricMinStandards} astrometric standards available in each CCD.

\textbf{Discussion:} The World Coordinate System for visits will be expressed in terms of a FITS Standard representation, which provides for named metadata to be interpreted as coefficients of one of a finite set of coordinate transformations.

\emph{Derived from Requirements:}

DMS-REQ-0030:
Generate WCS for Visit Images \newline

\subparagraph{Documenting Image Characterization}\hfill  % Force subsequent text onto new line

\label{DMS-L1-REQ-0012}
\textbf{ID:} DMS-L1-REQ-0012 (Priority: 1b)

\textbf{Specification:} The persisted format for Processed Visit Images shall be fully documented, and shall include a description of all image characterization data products.

\textbf{Discussion:} This will allow the community to use them to increase understanding of LSST images and derived LSST catalogs.

\emph{Derived from Requirements:}

DMS-REQ-0328:
Documenting Image Characterization \newline

\subparagraph{Processed Visit Image Content}\hfill  % Force subsequent text onto new line

\label{DMS-L1-REQ-0013}
\textbf{ID:} DMS-L1-REQ-0013 (Priority: 1a)

\textbf{Specification:} Processed visit images shall include the corrected science pixel array, an integer mask array where each bit-plane represents a logical statement about whether a particular detector pathology affects the pixel, a variance array which represents the expected variance in the corresponding science pixel, and a representation of the spatially varying PSF that applies over the extent of the science array. These images shall also contain metadata that map pixel to world (sky) coordinates (the WCS) as well as metadata from which photometric measurements can be derived.

\emph{Derived from Requirements:}

DMS-REQ-0072:
Processed Visit Image Content \newline

\paragraph{Difference Exposures}\hfill  % Force subsequent text onto new line

\label{DMS-L1-REQ-0005}
\textbf{ID:} DMS-L1-REQ-0005 (Priority: 1b)

\textbf{Specification:} The DMS shall create a Difference Exposure from each Processed Visit Image by subtracting a re-projected, scaled, PSF-matched Template Image in the same passband.

\textbf{Discussion:} Difference Exposures are not archived, and are retained for only a limited time to facilitate Alert processing. They can be re-generated for users on-demand.

\emph{Derived from Requirements:}

DMS-REQ-0010:
Difference Exposures \newline

\subparagraph{Difference Exposure Attributes}\hfill  % Force subsequent text onto new line

\label{DMS-L1-REQ-0006}
\textbf{ID:} DMS-L1-REQ-0006 (Priority: 1b)

\textbf{Specification:} For each Difference Exposure, the DMS shall store: the identify of the input exposures and related provenance information, and a set of metadata attributes including at least a representation of the PSF matching kernel used in the differencing.

\emph{Derived from Requirements:}

DMS-REQ-0074:
Difference Exposure Attributes \newline

\paragraph{Exposure Catalog}\hfill  % Force subsequent text onto new line

\label{DMS-L1-REQ-0015}
\textbf{ID:} DMS-L1-REQ-0015 (Priority: 1a)

\textbf{Specification:} The DMS shall create an Exposure Catalog containing information for each exposure that includes the exposure date/time and duration, properties of the filter used, dome and telescope pointing and orientation, status of calibration apparatus, airmass and zenith distance, telescope and dome status, environmental information, and information regarding each sensor including an ID, its location in the focal plane, electronic configuration, and WCS.

\emph{Derived from Requirements:}

DMS-REQ-0266:
Exposure Catalog \newline

\paragraph{DIASource Catalog}\hfill  % Force subsequent text onto new line

\label{DMS-L1-REQ-0016}
\textbf{ID:} DMS-L1-REQ-0016 (Priority: 1b)

\textbf{Specification:} The DMS shall construct a catalog of all Sources detected on Difference Exposures with SNR > \textbf{transSNR}. For each Difference Source (DIASource), the DMS shall be able to provide the identity of the Difference Exposure from which it was derived; the identity of the associated SSObject, if any; the identity of the parent Source from which this DIASource has been deblended, if any. The DMS shall also measure and record a set of attributes for each DIASource including at least: epoch of the observation, focal plane position centroid and error (pixel), sky position and associated error (radec), SNR of the detection; calibrated PS flux and associated error; likelihood of the observed data given the PS model; calibrated aperture flux and associated error; calibrated flux and associated error for a trailed source model, and length and angle of the trail; flux and associated parameters for a dipole model; parameters of an adaptive shape measurement and associated error; a measure of source extendedness; the estimated background at the position of the object in the template image with associated uncertainty; a measure of spuriousness; and flags indicating problems encountered while computing the aforementioned attributes. The DMS shall also determine and record measurements on the Calibrated exposure the following: calibrated flux and associated error for the source as measured on the Visit image.

\emph{Derived from Requirements:}

DMS-REQ-0269:
DIASource Catalog \newline

\paragraph{Faint DIASource Measurements}\hfill  % Force subsequent text onto new line

\label{DMS-L1-REQ-0017}
\textbf{ID:} DMS-L1-REQ-0017 (Priority: 2)

\textbf{Specification:} The DMS shall be able to measure and store DIASources fainter than \textbf{transSNR }that satisfy additional criteria. A limited number of such sources shall be made to enable monitoring of DIA quality.

\textbf{Discussion: }Some individual faint sources may be of high significance, such as a potentially hazardous asteroid.

\emph{Derived from Requirements:}

DMS-REQ-0270:
Faint DIASource Measurements \newline

\paragraph{Characterizing Variability}\hfill  % Force subsequent text onto new line

\label{DMS-L1-REQ-0021}
\textbf{ID:} DMS-L1-REQ-0021 (Priority: 1b)

\textbf{Specification:} For alert production, DIAObject variability characterization shall include data collected during the time period from the present to at least \textbf{diaCharacterizationCutoff} in the past.

\textbf{Discussion:} These measurements can come from the live L1 database. For level 1 processing during Data Release Production, all data should be used for characterization.

\emph{Derived from Requirements:}

DMS-REQ-0319:
Characterizing Variability \newline

\paragraph{DIAObject Catalog}\hfill  % Force subsequent text onto new line

\label{DMS-L1-REQ-0003}
\textbf{ID:} DMS-L1-REQ-0003 (Priority: 1b)

\textbf{Specification:} The DMS shall construct a catalog of all astrophysical objects identified through difference image analysis (DIAObjects). The DIAObject entries shall include metadata attributes including at least: a unique identifier; the identifiers of the\textbf{ diaNearbyObjMaxStar} nearest stars and \textbf{diaNearbyObjMaxGalaxy} nearest galaxies in the Object catalog lying within \textbf{diaNearbyObjRadius}, the probability that the DIAObject is the same as the nearby Object; and a set of DIAObject properties.

\emph{Derived from Requirements:}

DMS-REQ-0271:
DIAObject Catalog \newline

\subparagraph{DIAObject Attributes}\hfill  % Force subsequent text onto new line

\label{DMS-L1-REQ-0004}
\textbf{ID:} DMS-L1-REQ-0004 (Priority: 1b)

\textbf{Specification:} For each DIAObject the DMS shall store summary attributes including at least: sky position at the time of the observation; astrometric attributes including proper motion, parallax and related errors; point-source magnitude in each passband and related error; weighted mean forced-photometry flux and related error; periodic and non-periodic variability measures; and flags that encode special conditions encountered in measuring the above quantities.

\emph{Derived from Requirements:}

DMS-REQ-0272:
DIAObject Attributes \newline

\paragraph{SSObject Catalog}\hfill  % Force subsequent text onto new line

\label{DMS-L1-REQ-0018}
\textbf{ID:} DMS-L1-REQ-0018 (Priority: 2)

\textbf{Specification:} The DMS shall produce a catalog of all Solar System Objects (SSObjects) that have been identified via Moving Object Processing. The SSObject catalog shall include for each entry attributes including at least the following: Osculating orbital elements and associated uncertainties, minimum orbit intersection distance (MOID), mean absolute magnitude and slope parameter per band and associated errors, and flags that describe conditions of the description.

\textbf{Discussion: }The magnitude and angular velocity limits for identifying SSObjects are TBD. These limits may be driven more by computational resource constraints than by the raw reach of the collected data. The software may well be capable of exceeding the required limits, but at an unacceptable cost. The slope parameter will be poorly constrained until later in the survey. A baseline algorithm and acceptance criteria should be developed prior to verification.

\emph{Derived from Requirements:}

DMS-REQ-0273:
SSObject Catalog \newline

\paragraph{DIAForcedSource Catalog}\hfill  % Force subsequent text onto new line

\label{DMS-L1-REQ-0020}
\textbf{ID:} DMS-L1-REQ-0020 (Priority: 2)

\textbf{Specification:} The DMS shall create a DIAForcedSource Catalog, consisting of measured fluxes for entries in the DIAObject Catalog on Difference Exposures. Measurements for each forced-source shall include the DIAObject and visit IDs, the modeled flux and error (given fixed position, shape, and deblending parameters), and measurement quality flags.

\textbf{Discussion: }The large number of such forced sources makes it impractical to measure more attributes than are necessary to construct a light curve for variability studies.

\emph{Derived from Requirements:}

DMS-REQ-0317:
DIAForcedSource Catalog \newline

\paragraph{Matching DIASources to Objects}\hfill  % Force subsequent text onto new line

\label{DMS-L1-REQ-0022}
\textbf{ID:} DMS-L1-REQ-0022 (Priority: 1b)

\textbf{Specification:} A L1 DIASource to L2 Object positional cross-match table or database view shall be made available.

\textbf{Discussion:} Care should be taken to note that this is purely a cross-match based on separation on the sky and does not imply the DIASource and Object are physically the same.

\emph{Derived from Requirements:}

DMS-REQ-0324:
Matching DIASources to Objects \newline

\paragraph{Alert Content}\hfill  % Force subsequent text onto new line

\label{DMS-L1-REQ-0019}
\textbf{ID:} DMS-L1-REQ-0019 (Priority: 1b)

\textbf{Specification:} The DMS shall create an Alert for each detected DIASource, to be broadcast using community protocols, with content that includes: a unique Alert ID, the Level-1 database ID, the DIASource record that triggered the alert, the DIAObject (or SSObject) record, all previous DIASource records corresponding to the object (if any), and cut-outs of images (from both the template image and the difference image) of sufficient areal coverage to identify the DIASource and its immediate surroundings. These cutouts should include WCS, PSF, variance and mask information.

\textbf{Discussion: }The aim for the Alert content is to include sufficient information to be relatively self-contained, and to minimize the demand for follow-up queries of the Level-1 database. This approach will likely increase the speed and efficiency of down-stream object classifiers.

\emph{Derived from Requirements:}

DMS-REQ-0274:
Alert Content \newline

\paragraph{Level 1 Data Quality Report Definition}\hfill  % Force subsequent text onto new line

\label{DMS-L1-REQ-0014}
\textbf{ID:} DMS-L1-REQ-0014 (Priority: 1a)

\textbf{Specification:} The DMS shall produce a Level 1 Data Quality Report that contains indicators of data quality that result from running the DMS pipelines, including at least: Photometric zero point vs. time for each utilized filter; Sky brightness vs. time for each utilized filter; seeing vs. time for each utilized filter; PSF parameters vs. time for each utilized filter; detection efficiency for point sources vs. mag for each utilized filter.

\textbf{Discussion:} The seeing report is intended as a broad-brush measure of image quality.  The PSF parameters provide more detail, as they include asymmetries and field location dependence.

\emph{Derived from Requirements:}

DMS-REQ-0097:
Level 1 Data Quality Report Definition \newline

\subsubsection{Special Programs}

\paragraph{Level 1 Processing of Special Programs Data}\hfill  % Force subsequent text onto new line

\label{DMS-L1-REQ-0041}
\textbf{ID:} DMS-L1-REQ-0041 (Priority: 2)

\textbf{Specification:} All Level 1 processing from special programs shall be completed before data arrives from the following night's observations.

\emph{Derived from Requirements:}

DMS-REQ-0321:
Level 1 Processing of Special Programs Data \newline

\paragraph{Constraints on Level 1 Special Program Products Generation}\hfill  % Force subsequent text onto new line

\label{DMS-L1-REQ-0045}
\textbf{ID:} DMS-L1-REQ-0045 (Priority: 2)

\textbf{Specification: }The publishing of Level 1 data products from Special Programs shall be subject to the same performance requirements of the standard Level 1 system. In particular \textbf{L1PublicT} and \textbf{OTT1}.

\emph{Derived from Requirements:}

DMS-REQ-0344:
Constraints on Level 1 Special Program Products Generation \newline

\subsection{Productions}

\subsubsection{Alert Production}

\paragraph{Level-1 Production Completeness}\hfill  % Force subsequent text onto new line

\label{DMS-L1-REQ-0025}
\textbf{ID:} DMS-L1-REQ-0025 (Priority: 1b)

\textbf{Specification:} The DMS shall ensure that all images taken by the camera and marked for Level-1 processing are eventually retrieved, archived, and processed even in the event of connectivity failure between downstream Facilities.

\emph{Derived from Requirements:}

DMS-REQ-0284:
Level-1 Production Completeness \newline

\paragraph{Transient Alert Distribution}\hfill  % Force subsequent text onto new line

\label{DMS-L1-REQ-0001}
\textbf{ID:} DMS-L1-REQ-0001 (Priority: 1b)

\textbf{Specification:} Identified transient events shall be made available to end-users in the form of alerts, which shall be published to community alert distribution networks using community-standard protocols, to be determined during the LSST construction phase as community standards evolve.

\emph{Derived from Requirements:}

DMS-REQ-0002:
Transient Alert Distribution \newline

\paragraph{Alert Filtering Service}\hfill  % Force subsequent text onto new line

\label{DMS-L1-REQ-0030}
\textbf{ID:} DMS-L1-REQ-0030 (Priority: 2)

\textbf{Specification:} A basic, limited capacity, alert filtering service shall be provided that can be given user defined filters to reduce the alert stream to manageable levels.

\emph{Derived from Requirements:}

DMS-REQ-0342:
Alert Filtering Service \newline

\subparagraph{Pre-defined alert filters}\hfill  % Force subsequent text onto new line

\label{DMS-L1-REQ-0031}
\textbf{ID:} DMS-L1-REQ-0031 (Priority: 2)

\textbf{Specification: }Users of the LSST Alert Filtering Service shall be able to use a predefined set of simple filters.

\textbf{Discussion:} See LSR-REQ-0026

\emph{Derived from Requirements:}

DMS-REQ-0348:
Pre-defined alert filters \newline

\subparagraph{Performance Requirements for LSST Alert Filtering Service}\hfill  % Force subsequent text onto new line

\label{DMS-L1-REQ-0032}
\textbf{ID:} DMS-L1-REQ-0032 (Priority: 2)

\textbf{Specification:} The LSST alert filtering service shall support \textbf{numBrokerUsers} simultaneous users with each user allocated a bandwidth capable of receiving the equivalent of \textbf{numBrokerAlerts} alerts per visit.

\textbf{Discussion:} The constraint on number of alerts is specified for the full VOEvent alert content, but could also be satisfied by all alerts being received with minimal alert content.

\emph{Derived from Requirements:}

DMS-REQ-0343:
Performance Requirements for LSST Alert Filtering Service \newline

\paragraph{Level 1 Source Association}\hfill  % Force subsequent text onto new line

\label{DMS-L1-REQ-0026}
\textbf{ID:} DMS-L1-REQ-0026 (Priority: 1b)

\textbf{Specification:} The DMS shall associate clusters of DIASources detected on multiple visits taken at different times with either a DIAObject or an SSObject.

\textbf{Discussion: }The association will represent the underlying astrophysical phenomenon.

\emph{Derived from Requirements:}

DMS-REQ-0285:
Level 1 Source Association \newline

\paragraph{SSObject Precovery}\hfill  % Force subsequent text onto new line

\label{DMS-L1-REQ-0027}
\textbf{ID:} DMS-L1-REQ-0027 (Priority: 2)

\textbf{Specification:} Upon identifying a new SSObject, the DMS shall associate additional DIAObjects that are consistent with the orbital parameters (precovery), and update DIAObject entries so associated.

\emph{Derived from Requirements:}

DMS-REQ-0286:
SSObject Precovery \newline

\paragraph{DIASource Precovery}\hfill  % Force subsequent text onto new line

\label{DMS-L1-REQ-0028}
\textbf{ID:} DMS-L1-REQ-0028 (Priority: 1b)

\textbf{Specification:} For all DIASources not associated with either DIAObjects or SSObjects, the DMS shall perform forced photometry at the location of the new source (precovery) on all Difference Exposures obtained in the prior \textbf{precoveryWindow}, and make the results publicly available within \textbf{L1PublicT}.

\textbf{Discussion: }The \textbf{precoveryWindow }is intended to satisfy the most common scientific use cases (e.g., Supernovae), without placing an undue burden on the processing infrastructure.  For reasons of practicality and efficiency, \textbf{precoveryWindow }<= l\textbf{1CacheLifetime}.

\emph{Derived from Requirements:}

DMS-REQ-0287:
DIASource Precovery \newline

\paragraph{Use of External Orbit Catalogs}\hfill  % Force subsequent text onto new line

\label{DMS-L1-REQ-0029}
\textbf{ID:} DMS-L1-REQ-0029 (Priority: 2)

\textbf{Specification:} It shall be possible for DMS to make use of approved external catalogs and observations to improve the identification of SSObjects, and therefore increase the purity of the transient Alert stream in nightly processing.

\emph{Derived from Requirements:}

DMS-REQ-0288:
Use of External Orbit Catalogs \newline

\paragraph{Solar System Objects Available Within Specified Time}\hfill  % Force subsequent text onto new line

\label{DMS-L1-REQ-0023}
\textbf{ID:} DMS-L1-REQ-0023 (Priority: 1b)

\textbf{Specification:} Detected moving objects and associated metadata shall be available for public access in the DMS science data archive within time \textbf{L1PublicT }of their generation by the DMS.

\emph{Derived from Requirements:}

DMS-REQ-0089:
Solar System Objects Available Within Specified Time \newline

\paragraph{Generate Data Quality Report Within Specified Time}\hfill  % Force subsequent text onto new line

\label{DMS-L1-REQ-0024}
\textbf{ID:} DMS-L1-REQ-0024 (Priority: 1a)

\textbf{Specification:} The DMS shall generate a nightly Data Quality Report within time \textbf{dqReportComplTime }in both human-readable and machine-readable forms.

\textbf{Discussion:} The Report must be timely in order to evaluate whether changes to hardware, software, or procedures are needed for the following night's observing.

\emph{Derived from Requirements:}

DMS-REQ-0096:
Generate Data Quality Report Within Specified Time \newline

\subsection{Software}

\subsubsection{General Considerations}

\paragraph{Software Architecture to Enable Community Re-Use}\hfill  % Force subsequent text onto new line

\label{DMS-L1-REQ-0033}
\textbf{ID:} DMS-L1-REQ-0033 (Priority: 1b)

\textbf{Specification:} The DMS software architecture shall be designed to enable high throughput on high-performance compute platforms, while also enabling the use of science-specific algorithms by science users on commodity desktop compute platforms.

\textbf{Discussion: }The high data volume and short processing timeline for LSST Productions anticipates the use of high-performance compute infrastructure, while the need to make the science algorithms immediately applicable to science teams for Level-3 processing drives the need for easy interoperability with desktop compute environments.

\emph{Derived from Requirements:}

DMS-REQ-0308:
Software Architecture to Enable Community Re-Use \newline

\subsubsection{Applications Software}

\paragraph{Simulated Data}\hfill  % Force subsequent text onto new line

\label{DMS-L1-REQ-0034}
\textbf{ID:} DMS-L1-REQ-0034 (Priority: 1b)

\textbf{Specification:} The DMS shall provide the ability to inject artificial or simulated data into data products to assess the functional and temporal performance of the production processing software.

\emph{Derived from Requirements:}

DMS-REQ-0009:
Simulated Data \newline

\paragraph{Image Differencing}\hfill  % Force subsequent text onto new line

\label{DMS-L1-REQ-0035}
\textbf{ID:} DMS-L1-REQ-0035 (Priority: 1b)

\textbf{Specification:} The DMS shall provide software to perform image differencing, generating Difference Exposures from the comparison of single exposures and/or coadded images.

\emph{Derived from Requirements:}

DMS-REQ-0032:
Image Differencing \newline

\paragraph{Provide Source Detection Software}\hfill  % Force subsequent text onto new line

\label{DMS-L1-REQ-0036}
\textbf{ID:} DMS-L1-REQ-0036 (Priority: 1a)

\textbf{Specification:} The DMS shall provide software for the detection of sources in a calibrated image, which may be a Difference Image or a Co-Add image.

\emph{Derived from Requirements:}

DMS-REQ-0033:
Provide Source Detection Software \newline

\paragraph{Provide Calibrated Photometry}\hfill  % Force subsequent text onto new line

\label{DMS-L1-REQ-0038}
\textbf{ID:} DMS-L1-REQ-0038 (Priority: 1a)

\textbf{Specification:} The DMS shall provide calibrated photometry in each observed passband for all measured entities (e.g., DIASources, Sources, Objects), measuring the AB magnitude of the equivalent flat-SED source, above the atmosphere. Fluxes, possibly in jansky, shall be calculated for all measured entities.

\textbf{Discussion: }Note that the SED is only assumed to be flat within the passband of the measurement.

\emph{Derived from Requirements:}

DMS-REQ-0043:
Provide Calibrated Photometry \newline

\paragraph{Provide Astrometric Model}\hfill  % Force subsequent text onto new line

\label{DMS-L1-REQ-0037}
\textbf{ID:} DMS-L1-REQ-0037 (Priority: 1b)

\textbf{Specification:} An astrometric model shall be provided for every Object and DIAObject which specifies at least the proper motion and parallax, and the estimated uncertainties on these quantities.

\emph{Derived from Requirements:}

DMS-REQ-0042:
Provide Astrometric Model \newline

\paragraph{Enable a Range of Shape Measurement Approaches}\hfill  % Force subsequent text onto new line

\label{DMS-L1-REQ-0039}
\textbf{ID:} DMS-L1-REQ-0039 (Priority: 1b)

\textbf{Specification:} The DMS shall provide for the use of a variety of shape models on multiple kinds of input data to measure sources: measurement on coadds; measurement on coadds using information (e.g., PSFs) extracted from the individual exposures; measurement based on all the information from the individual Exposures simultaneously.

\textbf{Discussion: }The most appropriate measurement model to apply depends upon the nature of the composite source.

\emph{Derived from Requirements:}

DMS-REQ-0052:
Enable a Range of Shape Measurement Approaches \newline

\subsection{Facilities}

\subsubsection{Data Archive}

\paragraph{Level 1 Data Product Access}\hfill  % Force subsequent text onto new line

\label{DMS-L1-REQ-0040}
\textbf{ID:} DMS-L1-REQ-0040 (Priority: 1b)

\textbf{Specification:} The DMS shall maintain a "live" Level 1 Database for query by science users, updated as a result of Alert Production processing.

\emph{Derived from Requirements:}

DMS-REQ-0312:
Level 1 Data Product Access \newline

\addendum

\bibliography{lsst,refs_ads}

\end{document}
